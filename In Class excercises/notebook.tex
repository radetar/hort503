
% Default to the notebook output style

    


% Inherit from the specified cell style.




    
\documentclass[11pt]{article}

    
    
    \usepackage[T1]{fontenc}
    % Nicer default font (+ math font) than Computer Modern for most use cases
    \usepackage{mathpazo}

    % Basic figure setup, for now with no caption control since it's done
    % automatically by Pandoc (which extracts ![](path) syntax from Markdown).
    \usepackage{graphicx}
    % We will generate all images so they have a width \maxwidth. This means
    % that they will get their normal width if they fit onto the page, but
    % are scaled down if they would overflow the margins.
    \makeatletter
    \def\maxwidth{\ifdim\Gin@nat@width>\linewidth\linewidth
    \else\Gin@nat@width\fi}
    \makeatother
    \let\Oldincludegraphics\includegraphics
    % Set max figure width to be 80% of text width, for now hardcoded.
    \renewcommand{\includegraphics}[1]{\Oldincludegraphics[width=.8\maxwidth]{#1}}
    % Ensure that by default, figures have no caption (until we provide a
    % proper Figure object with a Caption API and a way to capture that
    % in the conversion process - todo).
    \usepackage{caption}
    \DeclareCaptionLabelFormat{nolabel}{}
    \captionsetup{labelformat=nolabel}

    \usepackage{adjustbox} % Used to constrain images to a maximum size 
    \usepackage{xcolor} % Allow colors to be defined
    \usepackage{enumerate} % Needed for markdown enumerations to work
    \usepackage{geometry} % Used to adjust the document margins
    \usepackage{amsmath} % Equations
    \usepackage{amssymb} % Equations
    \usepackage{textcomp} % defines textquotesingle
    % Hack from http://tex.stackexchange.com/a/47451/13684:
    \AtBeginDocument{%
        \def\PYZsq{\textquotesingle}% Upright quotes in Pygmentized code
    }
    \usepackage{upquote} % Upright quotes for verbatim code
    \usepackage{eurosym} % defines \euro
    \usepackage[mathletters]{ucs} % Extended unicode (utf-8) support
    \usepackage[utf8x]{inputenc} % Allow utf-8 characters in the tex document
    \usepackage{fancyvrb} % verbatim replacement that allows latex
    \usepackage{grffile} % extends the file name processing of package graphics 
                         % to support a larger range 
    % The hyperref package gives us a pdf with properly built
    % internal navigation ('pdf bookmarks' for the table of contents,
    % internal cross-reference links, web links for URLs, etc.)
    \usepackage{hyperref}
    \usepackage{longtable} % longtable support required by pandoc >1.10
    \usepackage{booktabs}  % table support for pandoc > 1.12.2
    \usepackage[inline]{enumitem} % IRkernel/repr support (it uses the enumerate* environment)
    \usepackage[normalem]{ulem} % ulem is needed to support strikethroughs (\sout)
                                % normalem makes italics be italics, not underlines
    

    
    
    % Colors for the hyperref package
    \definecolor{urlcolor}{rgb}{0,.145,.698}
    \definecolor{linkcolor}{rgb}{.71,0.21,0.01}
    \definecolor{citecolor}{rgb}{.12,.54,.11}

    % ANSI colors
    \definecolor{ansi-black}{HTML}{3E424D}
    \definecolor{ansi-black-intense}{HTML}{282C36}
    \definecolor{ansi-red}{HTML}{E75C58}
    \definecolor{ansi-red-intense}{HTML}{B22B31}
    \definecolor{ansi-green}{HTML}{00A250}
    \definecolor{ansi-green-intense}{HTML}{007427}
    \definecolor{ansi-yellow}{HTML}{DDB62B}
    \definecolor{ansi-yellow-intense}{HTML}{B27D12}
    \definecolor{ansi-blue}{HTML}{208FFB}
    \definecolor{ansi-blue-intense}{HTML}{0065CA}
    \definecolor{ansi-magenta}{HTML}{D160C4}
    \definecolor{ansi-magenta-intense}{HTML}{A03196}
    \definecolor{ansi-cyan}{HTML}{60C6C8}
    \definecolor{ansi-cyan-intense}{HTML}{258F8F}
    \definecolor{ansi-white}{HTML}{C5C1B4}
    \definecolor{ansi-white-intense}{HTML}{A1A6B2}

    % commands and environments needed by pandoc snippets
    % extracted from the output of `pandoc -s`
    \providecommand{\tightlist}{%
      \setlength{\itemsep}{0pt}\setlength{\parskip}{0pt}}
    \DefineVerbatimEnvironment{Highlighting}{Verbatim}{commandchars=\\\{\}}
    % Add ',fontsize=\small' for more characters per line
    \newenvironment{Shaded}{}{}
    \newcommand{\KeywordTok}[1]{\textcolor[rgb]{0.00,0.44,0.13}{\textbf{{#1}}}}
    \newcommand{\DataTypeTok}[1]{\textcolor[rgb]{0.56,0.13,0.00}{{#1}}}
    \newcommand{\DecValTok}[1]{\textcolor[rgb]{0.25,0.63,0.44}{{#1}}}
    \newcommand{\BaseNTok}[1]{\textcolor[rgb]{0.25,0.63,0.44}{{#1}}}
    \newcommand{\FloatTok}[1]{\textcolor[rgb]{0.25,0.63,0.44}{{#1}}}
    \newcommand{\CharTok}[1]{\textcolor[rgb]{0.25,0.44,0.63}{{#1}}}
    \newcommand{\StringTok}[1]{\textcolor[rgb]{0.25,0.44,0.63}{{#1}}}
    \newcommand{\CommentTok}[1]{\textcolor[rgb]{0.38,0.63,0.69}{\textit{{#1}}}}
    \newcommand{\OtherTok}[1]{\textcolor[rgb]{0.00,0.44,0.13}{{#1}}}
    \newcommand{\AlertTok}[1]{\textcolor[rgb]{1.00,0.00,0.00}{\textbf{{#1}}}}
    \newcommand{\FunctionTok}[1]{\textcolor[rgb]{0.02,0.16,0.49}{{#1}}}
    \newcommand{\RegionMarkerTok}[1]{{#1}}
    \newcommand{\ErrorTok}[1]{\textcolor[rgb]{1.00,0.00,0.00}{\textbf{{#1}}}}
    \newcommand{\NormalTok}[1]{{#1}}
    
    % Additional commands for more recent versions of Pandoc
    \newcommand{\ConstantTok}[1]{\textcolor[rgb]{0.53,0.00,0.00}{{#1}}}
    \newcommand{\SpecialCharTok}[1]{\textcolor[rgb]{0.25,0.44,0.63}{{#1}}}
    \newcommand{\VerbatimStringTok}[1]{\textcolor[rgb]{0.25,0.44,0.63}{{#1}}}
    \newcommand{\SpecialStringTok}[1]{\textcolor[rgb]{0.73,0.40,0.53}{{#1}}}
    \newcommand{\ImportTok}[1]{{#1}}
    \newcommand{\DocumentationTok}[1]{\textcolor[rgb]{0.73,0.13,0.13}{\textit{{#1}}}}
    \newcommand{\AnnotationTok}[1]{\textcolor[rgb]{0.38,0.63,0.69}{\textbf{\textit{{#1}}}}}
    \newcommand{\CommentVarTok}[1]{\textcolor[rgb]{0.38,0.63,0.69}{\textbf{\textit{{#1}}}}}
    \newcommand{\VariableTok}[1]{\textcolor[rgb]{0.10,0.09,0.49}{{#1}}}
    \newcommand{\ControlFlowTok}[1]{\textcolor[rgb]{0.00,0.44,0.13}{\textbf{{#1}}}}
    \newcommand{\OperatorTok}[1]{\textcolor[rgb]{0.40,0.40,0.40}{{#1}}}
    \newcommand{\BuiltInTok}[1]{{#1}}
    \newcommand{\ExtensionTok}[1]{{#1}}
    \newcommand{\PreprocessorTok}[1]{\textcolor[rgb]{0.74,0.48,0.00}{{#1}}}
    \newcommand{\AttributeTok}[1]{\textcolor[rgb]{0.49,0.56,0.16}{{#1}}}
    \newcommand{\InformationTok}[1]{\textcolor[rgb]{0.38,0.63,0.69}{\textbf{\textit{{#1}}}}}
    \newcommand{\WarningTok}[1]{\textcolor[rgb]{0.38,0.63,0.69}{\textbf{\textit{{#1}}}}}
    
    
    % Define a nice break command that doesn't care if a line doesn't already
    % exist.
    \def\br{\hspace*{\fill} \\* }
    % Math Jax compatability definitions
    \def\gt{>}
    \def\lt{<}
    % Document parameters
    \title{assignment10\_tyler}
    
    
    

    % Pygments definitions
    
\makeatletter
\def\PY@reset{\let\PY@it=\relax \let\PY@bf=\relax%
    \let\PY@ul=\relax \let\PY@tc=\relax%
    \let\PY@bc=\relax \let\PY@ff=\relax}
\def\PY@tok#1{\csname PY@tok@#1\endcsname}
\def\PY@toks#1+{\ifx\relax#1\empty\else%
    \PY@tok{#1}\expandafter\PY@toks\fi}
\def\PY@do#1{\PY@bc{\PY@tc{\PY@ul{%
    \PY@it{\PY@bf{\PY@ff{#1}}}}}}}
\def\PY#1#2{\PY@reset\PY@toks#1+\relax+\PY@do{#2}}

\expandafter\def\csname PY@tok@w\endcsname{\def\PY@tc##1{\textcolor[rgb]{0.73,0.73,0.73}{##1}}}
\expandafter\def\csname PY@tok@c\endcsname{\let\PY@it=\textit\def\PY@tc##1{\textcolor[rgb]{0.25,0.50,0.50}{##1}}}
\expandafter\def\csname PY@tok@cp\endcsname{\def\PY@tc##1{\textcolor[rgb]{0.74,0.48,0.00}{##1}}}
\expandafter\def\csname PY@tok@k\endcsname{\let\PY@bf=\textbf\def\PY@tc##1{\textcolor[rgb]{0.00,0.50,0.00}{##1}}}
\expandafter\def\csname PY@tok@kp\endcsname{\def\PY@tc##1{\textcolor[rgb]{0.00,0.50,0.00}{##1}}}
\expandafter\def\csname PY@tok@kt\endcsname{\def\PY@tc##1{\textcolor[rgb]{0.69,0.00,0.25}{##1}}}
\expandafter\def\csname PY@tok@o\endcsname{\def\PY@tc##1{\textcolor[rgb]{0.40,0.40,0.40}{##1}}}
\expandafter\def\csname PY@tok@ow\endcsname{\let\PY@bf=\textbf\def\PY@tc##1{\textcolor[rgb]{0.67,0.13,1.00}{##1}}}
\expandafter\def\csname PY@tok@nb\endcsname{\def\PY@tc##1{\textcolor[rgb]{0.00,0.50,0.00}{##1}}}
\expandafter\def\csname PY@tok@nf\endcsname{\def\PY@tc##1{\textcolor[rgb]{0.00,0.00,1.00}{##1}}}
\expandafter\def\csname PY@tok@nc\endcsname{\let\PY@bf=\textbf\def\PY@tc##1{\textcolor[rgb]{0.00,0.00,1.00}{##1}}}
\expandafter\def\csname PY@tok@nn\endcsname{\let\PY@bf=\textbf\def\PY@tc##1{\textcolor[rgb]{0.00,0.00,1.00}{##1}}}
\expandafter\def\csname PY@tok@ne\endcsname{\let\PY@bf=\textbf\def\PY@tc##1{\textcolor[rgb]{0.82,0.25,0.23}{##1}}}
\expandafter\def\csname PY@tok@nv\endcsname{\def\PY@tc##1{\textcolor[rgb]{0.10,0.09,0.49}{##1}}}
\expandafter\def\csname PY@tok@no\endcsname{\def\PY@tc##1{\textcolor[rgb]{0.53,0.00,0.00}{##1}}}
\expandafter\def\csname PY@tok@nl\endcsname{\def\PY@tc##1{\textcolor[rgb]{0.63,0.63,0.00}{##1}}}
\expandafter\def\csname PY@tok@ni\endcsname{\let\PY@bf=\textbf\def\PY@tc##1{\textcolor[rgb]{0.60,0.60,0.60}{##1}}}
\expandafter\def\csname PY@tok@na\endcsname{\def\PY@tc##1{\textcolor[rgb]{0.49,0.56,0.16}{##1}}}
\expandafter\def\csname PY@tok@nt\endcsname{\let\PY@bf=\textbf\def\PY@tc##1{\textcolor[rgb]{0.00,0.50,0.00}{##1}}}
\expandafter\def\csname PY@tok@nd\endcsname{\def\PY@tc##1{\textcolor[rgb]{0.67,0.13,1.00}{##1}}}
\expandafter\def\csname PY@tok@s\endcsname{\def\PY@tc##1{\textcolor[rgb]{0.73,0.13,0.13}{##1}}}
\expandafter\def\csname PY@tok@sd\endcsname{\let\PY@it=\textit\def\PY@tc##1{\textcolor[rgb]{0.73,0.13,0.13}{##1}}}
\expandafter\def\csname PY@tok@si\endcsname{\let\PY@bf=\textbf\def\PY@tc##1{\textcolor[rgb]{0.73,0.40,0.53}{##1}}}
\expandafter\def\csname PY@tok@se\endcsname{\let\PY@bf=\textbf\def\PY@tc##1{\textcolor[rgb]{0.73,0.40,0.13}{##1}}}
\expandafter\def\csname PY@tok@sr\endcsname{\def\PY@tc##1{\textcolor[rgb]{0.73,0.40,0.53}{##1}}}
\expandafter\def\csname PY@tok@ss\endcsname{\def\PY@tc##1{\textcolor[rgb]{0.10,0.09,0.49}{##1}}}
\expandafter\def\csname PY@tok@sx\endcsname{\def\PY@tc##1{\textcolor[rgb]{0.00,0.50,0.00}{##1}}}
\expandafter\def\csname PY@tok@m\endcsname{\def\PY@tc##1{\textcolor[rgb]{0.40,0.40,0.40}{##1}}}
\expandafter\def\csname PY@tok@gh\endcsname{\let\PY@bf=\textbf\def\PY@tc##1{\textcolor[rgb]{0.00,0.00,0.50}{##1}}}
\expandafter\def\csname PY@tok@gu\endcsname{\let\PY@bf=\textbf\def\PY@tc##1{\textcolor[rgb]{0.50,0.00,0.50}{##1}}}
\expandafter\def\csname PY@tok@gd\endcsname{\def\PY@tc##1{\textcolor[rgb]{0.63,0.00,0.00}{##1}}}
\expandafter\def\csname PY@tok@gi\endcsname{\def\PY@tc##1{\textcolor[rgb]{0.00,0.63,0.00}{##1}}}
\expandafter\def\csname PY@tok@gr\endcsname{\def\PY@tc##1{\textcolor[rgb]{1.00,0.00,0.00}{##1}}}
\expandafter\def\csname PY@tok@ge\endcsname{\let\PY@it=\textit}
\expandafter\def\csname PY@tok@gs\endcsname{\let\PY@bf=\textbf}
\expandafter\def\csname PY@tok@gp\endcsname{\let\PY@bf=\textbf\def\PY@tc##1{\textcolor[rgb]{0.00,0.00,0.50}{##1}}}
\expandafter\def\csname PY@tok@go\endcsname{\def\PY@tc##1{\textcolor[rgb]{0.53,0.53,0.53}{##1}}}
\expandafter\def\csname PY@tok@gt\endcsname{\def\PY@tc##1{\textcolor[rgb]{0.00,0.27,0.87}{##1}}}
\expandafter\def\csname PY@tok@err\endcsname{\def\PY@bc##1{\setlength{\fboxsep}{0pt}\fcolorbox[rgb]{1.00,0.00,0.00}{1,1,1}{\strut ##1}}}
\expandafter\def\csname PY@tok@kc\endcsname{\let\PY@bf=\textbf\def\PY@tc##1{\textcolor[rgb]{0.00,0.50,0.00}{##1}}}
\expandafter\def\csname PY@tok@kd\endcsname{\let\PY@bf=\textbf\def\PY@tc##1{\textcolor[rgb]{0.00,0.50,0.00}{##1}}}
\expandafter\def\csname PY@tok@kn\endcsname{\let\PY@bf=\textbf\def\PY@tc##1{\textcolor[rgb]{0.00,0.50,0.00}{##1}}}
\expandafter\def\csname PY@tok@kr\endcsname{\let\PY@bf=\textbf\def\PY@tc##1{\textcolor[rgb]{0.00,0.50,0.00}{##1}}}
\expandafter\def\csname PY@tok@bp\endcsname{\def\PY@tc##1{\textcolor[rgb]{0.00,0.50,0.00}{##1}}}
\expandafter\def\csname PY@tok@fm\endcsname{\def\PY@tc##1{\textcolor[rgb]{0.00,0.00,1.00}{##1}}}
\expandafter\def\csname PY@tok@vc\endcsname{\def\PY@tc##1{\textcolor[rgb]{0.10,0.09,0.49}{##1}}}
\expandafter\def\csname PY@tok@vg\endcsname{\def\PY@tc##1{\textcolor[rgb]{0.10,0.09,0.49}{##1}}}
\expandafter\def\csname PY@tok@vi\endcsname{\def\PY@tc##1{\textcolor[rgb]{0.10,0.09,0.49}{##1}}}
\expandafter\def\csname PY@tok@vm\endcsname{\def\PY@tc##1{\textcolor[rgb]{0.10,0.09,0.49}{##1}}}
\expandafter\def\csname PY@tok@sa\endcsname{\def\PY@tc##1{\textcolor[rgb]{0.73,0.13,0.13}{##1}}}
\expandafter\def\csname PY@tok@sb\endcsname{\def\PY@tc##1{\textcolor[rgb]{0.73,0.13,0.13}{##1}}}
\expandafter\def\csname PY@tok@sc\endcsname{\def\PY@tc##1{\textcolor[rgb]{0.73,0.13,0.13}{##1}}}
\expandafter\def\csname PY@tok@dl\endcsname{\def\PY@tc##1{\textcolor[rgb]{0.73,0.13,0.13}{##1}}}
\expandafter\def\csname PY@tok@s2\endcsname{\def\PY@tc##1{\textcolor[rgb]{0.73,0.13,0.13}{##1}}}
\expandafter\def\csname PY@tok@sh\endcsname{\def\PY@tc##1{\textcolor[rgb]{0.73,0.13,0.13}{##1}}}
\expandafter\def\csname PY@tok@s1\endcsname{\def\PY@tc##1{\textcolor[rgb]{0.73,0.13,0.13}{##1}}}
\expandafter\def\csname PY@tok@mb\endcsname{\def\PY@tc##1{\textcolor[rgb]{0.40,0.40,0.40}{##1}}}
\expandafter\def\csname PY@tok@mf\endcsname{\def\PY@tc##1{\textcolor[rgb]{0.40,0.40,0.40}{##1}}}
\expandafter\def\csname PY@tok@mh\endcsname{\def\PY@tc##1{\textcolor[rgb]{0.40,0.40,0.40}{##1}}}
\expandafter\def\csname PY@tok@mi\endcsname{\def\PY@tc##1{\textcolor[rgb]{0.40,0.40,0.40}{##1}}}
\expandafter\def\csname PY@tok@il\endcsname{\def\PY@tc##1{\textcolor[rgb]{0.40,0.40,0.40}{##1}}}
\expandafter\def\csname PY@tok@mo\endcsname{\def\PY@tc##1{\textcolor[rgb]{0.40,0.40,0.40}{##1}}}
\expandafter\def\csname PY@tok@ch\endcsname{\let\PY@it=\textit\def\PY@tc##1{\textcolor[rgb]{0.25,0.50,0.50}{##1}}}
\expandafter\def\csname PY@tok@cm\endcsname{\let\PY@it=\textit\def\PY@tc##1{\textcolor[rgb]{0.25,0.50,0.50}{##1}}}
\expandafter\def\csname PY@tok@cpf\endcsname{\let\PY@it=\textit\def\PY@tc##1{\textcolor[rgb]{0.25,0.50,0.50}{##1}}}
\expandafter\def\csname PY@tok@c1\endcsname{\let\PY@it=\textit\def\PY@tc##1{\textcolor[rgb]{0.25,0.50,0.50}{##1}}}
\expandafter\def\csname PY@tok@cs\endcsname{\let\PY@it=\textit\def\PY@tc##1{\textcolor[rgb]{0.25,0.50,0.50}{##1}}}

\def\PYZbs{\char`\\}
\def\PYZus{\char`\_}
\def\PYZob{\char`\{}
\def\PYZcb{\char`\}}
\def\PYZca{\char`\^}
\def\PYZam{\char`\&}
\def\PYZlt{\char`\<}
\def\PYZgt{\char`\>}
\def\PYZsh{\char`\#}
\def\PYZpc{\char`\%}
\def\PYZdl{\char`\$}
\def\PYZhy{\char`\-}
\def\PYZsq{\char`\'}
\def\PYZdq{\char`\"}
\def\PYZti{\char`\~}
% for compatibility with earlier versions
\def\PYZat{@}
\def\PYZlb{[}
\def\PYZrb{]}
\makeatother


    % Exact colors from NB
    \definecolor{incolor}{rgb}{0.0, 0.0, 0.5}
    \definecolor{outcolor}{rgb}{0.545, 0.0, 0.0}



    
    % Prevent overflowing lines due to hard-to-break entities
    \sloppy 
    % Setup hyperref package
    \hypersetup{
      breaklinks=true,  % so long urls are correctly broken across lines
      colorlinks=true,
      urlcolor=urlcolor,
      linkcolor=linkcolor,
      citecolor=citecolor,
      }
    % Slightly bigger margins than the latex defaults
    
    \geometry{verbose,tmargin=1in,bmargin=1in,lmargin=1in,rmargin=1in}
    
    

    \begin{document}
    
    
    \maketitle
    
    

    
    \section{Part 1}\label{part-1}

\begin{quote}
Complete the online tutorial at
https://machinelearningmastery.com/machine-learning-in-python-step-by-step/.
\end{quote}

I do a few things differently than the author.

\begin{center}\rule{0.5\linewidth}{\linethickness}\end{center}

    \subsubsection{Package Imports}\label{package-imports}

I will import all my packages in the cell below.

    \begin{Verbatim}[commandchars=\\\{\}]
{\color{incolor}In [{\color{incolor}1}]:} \PY{k+kn}{import} \PY{n+nn}{sys}
        \PY{k+kn}{import} \PY{n+nn}{scipy}
        \PY{k+kn}{import} \PY{n+nn}{numpy} \PY{k}{as} \PY{n+nn}{np}
        \PY{k+kn}{import} \PY{n+nn}{seaborn} \PY{k}{as} \PY{n+nn}{sns}
        \PY{k+kn}{import} \PY{n+nn}{matplotlib}\PY{n+nn}{.}\PY{n+nn}{pyplot} \PY{k}{as} \PY{n+nn}{plt}
        \PY{k+kn}{import} \PY{n+nn}{pandas} \PY{k}{as} \PY{n+nn}{pd}
        
        \PY{c+c1}{\PYZsh{} I do not want to ommit the package name space when I use the functions within.}
        \PY{c+c1}{\PYZsh{} Some packages require more specific imports as they do not automatically}
        \PY{c+c1}{\PYZsh{} import subpackages.}
        \PY{k+kn}{import} \PY{n+nn}{sklearn} \PY{k}{as} \PY{n+nn}{sk}
        
        \PY{c+c1}{\PYZsh{} The data the author uses can be imported locally from sklearn, it does not}
        \PY{c+c1}{\PYZsh{} need to be downloaded.}
        \PY{k+kn}{from} \PY{n+nn}{sklearn} \PY{k}{import} \PY{n}{datasets}
        \PY{k+kn}{import} \PY{n+nn}{sklearn}\PY{n+nn}{.}\PY{n+nn}{model\PYZus{}selection}
\end{Verbatim}


    \begin{Verbatim}[commandchars=\\\{\}]
{\color{incolor}In [{\color{incolor}2}]:} \PY{c+c1}{\PYZsh{} sns.set\PYZus{}style(\PYZdq{}whitegrid\PYZdq{})}
        \PY{c+c1}{\PYZsh{} plt.style.use(\PYZsq{}dark\PYZus{}background\PYZsq{})}
\end{Verbatim}


    \begin{center}\rule{0.5\linewidth}{\linethickness}\end{center}

    \subsection{Start Python and Check
Versions}\label{start-python-and-check-versions}

\begin{quote}
It is a good idea to make sure your Python environment was installed
successfully and is working as expected.

The script below will help you test out your environment. It imports
each library required in this tutorial and prints the version.
\end{quote}

You can view the packages you are working with and their version from
the terminal with \texttt{pip\ freeze}.

    \begin{Verbatim}[commandchars=\\\{\}]
{\color{incolor}In [{\color{incolor}27}]:} \PY{c+c1}{\PYZsh{} It is cool to pull data like this, but there is no reason to in this case.}
         \PY{c+c1}{\PYZsh{} url = \PYZdq{}https://archive.ics.uci.edu/ml/machine\PYZhy{}learning\PYZhy{}databases/iris/iris.data\PYZdq{}}
         \PY{c+c1}{\PYZsh{} names = [\PYZsq{}sepal\PYZhy{}length\PYZsq{}, \PYZsq{}sepal\PYZhy{}width\PYZsq{}, \PYZsq{}petal\PYZhy{}length\PYZsq{}, \PYZsq{}petal\PYZhy{}width\PYZsq{}, \PYZsq{}class\PYZsq{}]}
         \PY{c+c1}{\PYZsh{} dataset = pd.read\PYZus{}csv(url, names=names)}
         \PY{n}{raw\PYZus{}dataset} \PY{o}{=} \PY{n}{datasets}\PY{o}{.}\PY{n}{load\PYZus{}iris}\PY{p}{(}\PY{p}{)}
         \PY{n}{dataset} \PY{o}{=} \PY{n}{pd}\PY{o}{.}\PY{n}{DataFrame}\PY{p}{(}\PY{n}{raw\PYZus{}dataset}\PY{o}{.}\PY{n}{data}\PY{p}{,} \PY{n}{columns}\PY{o}{=}\PY{n}{raw\PYZus{}dataset}\PY{o}{.}\PY{n}{feature\PYZus{}names}\PY{p}{)}
\end{Verbatim}


    \begin{Verbatim}[commandchars=\\\{\}]
{\color{incolor}In [{\color{incolor}44}]:} \PY{c+c1}{\PYZsh{} dataset.head()}
\end{Verbatim}


    \begin{Verbatim}[commandchars=\\\{\}]
{\color{incolor}In [{\color{incolor}29}]:} \PY{c+c1}{\PYZsh{} I am missing the classification column.}
         \PY{n}{dataset}\PY{p}{[}\PY{l+s+s1}{\PYZsq{}}\PY{l+s+s1}{class}\PY{l+s+s1}{\PYZsq{}}\PY{p}{]} \PY{o}{=} \PY{p}{[}\PY{n}{raw\PYZus{}dataset}\PY{o}{.}\PY{n}{target\PYZus{}names}\PY{p}{[}\PY{n}{x}\PY{p}{]} \PY{k}{for} \PY{n}{x} \PY{o+ow}{in} \PY{n}{raw\PYZus{}dataset}\PY{o}{.}\PY{n}{target}\PY{p}{]}
\end{Verbatim}


    \begin{Verbatim}[commandchars=\\\{\}]
{\color{incolor}In [{\color{incolor}31}]:} \PY{n}{dataset}\PY{o}{.}\PY{n}{head}\PY{p}{(}\PY{p}{)}
\end{Verbatim}


\begin{Verbatim}[commandchars=\\\{\}]
{\color{outcolor}Out[{\color{outcolor}31}]:}    sepal length (cm)  sepal width (cm)  petal length (cm)  petal width (cm)  \textbackslash{}
         0                5.1               3.5                1.4               0.2   
         1                4.9               3.0                1.4               0.2   
         2                4.7               3.2                1.3               0.2   
         3                4.6               3.1                1.5               0.2   
         4                5.0               3.6                1.4               0.2   
         
             class  
         0  setosa  
         1  setosa  
         2  setosa  
         3  setosa  
         4  setosa  
\end{Verbatim}
            
    \begin{Verbatim}[commandchars=\\\{\}]
{\color{incolor}In [{\color{incolor}32}]:} \PY{n}{dataset}\PY{o}{.}\PY{n}{describe}\PY{p}{(}\PY{p}{)}
\end{Verbatim}


\begin{Verbatim}[commandchars=\\\{\}]
{\color{outcolor}Out[{\color{outcolor}32}]:}        sepal length (cm)  sepal width (cm)  petal length (cm)  \textbackslash{}
         count         150.000000        150.000000         150.000000   
         mean            5.843333          3.054000           3.758667   
         std             0.828066          0.433594           1.764420   
         min             4.300000          2.000000           1.000000   
         25\%             5.100000          2.800000           1.600000   
         50\%             5.800000          3.000000           4.350000   
         75\%             6.400000          3.300000           5.100000   
         max             7.900000          4.400000           6.900000   
         
                petal width (cm)  
         count        150.000000  
         mean           1.198667  
         std            0.763161  
         min            0.100000  
         25\%            0.300000  
         50\%            1.300000  
         75\%            1.800000  
         max            2.500000  
\end{Verbatim}
            
    \begin{Verbatim}[commandchars=\\\{\}]
{\color{incolor}In [{\color{incolor}33}]:} \PY{n}{dataset}\PY{o}{.}\PY{n}{groupby}\PY{p}{(}\PY{l+s+s1}{\PYZsq{}}\PY{l+s+s1}{class}\PY{l+s+s1}{\PYZsq{}}\PY{p}{)}\PY{o}{.}\PY{n}{size}\PY{p}{(}\PY{p}{)}
\end{Verbatim}


\begin{Verbatim}[commandchars=\\\{\}]
{\color{outcolor}Out[{\color{outcolor}33}]:} class
         setosa        50
         versicolor    50
         virginica     50
         dtype: int64
\end{Verbatim}
            
    \begin{Verbatim}[commandchars=\\\{\}]
{\color{incolor}In [{\color{incolor}34}]:} \PY{n}{dataset}\PY{o}{.}\PY{n}{plot}\PY{p}{(}
             \PY{n}{kind}\PY{o}{=}\PY{l+s+s1}{\PYZsq{}}\PY{l+s+s1}{box}\PY{l+s+s1}{\PYZsq{}}\PY{p}{,} 
             \PY{n}{subplots}\PY{o}{=}\PY{k+kc}{True}\PY{p}{,} 
         \PY{c+c1}{\PYZsh{}     layout=(2,2), }
             \PY{n}{sharex}\PY{o}{=}\PY{k+kc}{False}\PY{p}{,} 
             \PY{n}{sharey}\PY{o}{=}\PY{k+kc}{True}\PY{p}{,}     \PY{c+c1}{\PYZsh{} Shared the y\PYZhy{}axis so the plots are comparable.}
             \PY{n}{figsize}\PY{o}{=}\PY{p}{(}\PY{l+m+mi}{10}\PY{p}{,}\PY{l+m+mi}{10}\PY{p}{)}  \PY{c+c1}{\PYZsh{} Enlarged the figure.}
         \PY{p}{)}
         \PY{n}{plt}\PY{o}{.}\PY{n}{show}\PY{p}{(}\PY{p}{)}
\end{Verbatim}


    \begin{center}
    \adjustimage{max size={0.9\linewidth}{0.9\paperheight}}{output_12_0.png}
    \end{center}
    { \hspace*{\fill} \\}
    
    \begin{Verbatim}[commandchars=\\\{\}]
{\color{incolor}In [{\color{incolor}35}]:} \PY{n}{dataset}\PY{o}{.}\PY{n}{hist}\PY{p}{(}
             \PY{n}{figsize}\PY{o}{=}\PY{p}{(}\PY{l+m+mi}{10}\PY{p}{,}\PY{l+m+mi}{10}\PY{p}{)}
         \PY{p}{)}
         \PY{n}{plt}\PY{o}{.}\PY{n}{show}\PY{p}{(}\PY{p}{)}
\end{Verbatim}


    \begin{center}
    \adjustimage{max size={0.9\linewidth}{0.9\paperheight}}{output_13_0.png}
    \end{center}
    { \hspace*{\fill} \\}
    
    \begin{Verbatim}[commandchars=\\\{\}]
{\color{incolor}In [{\color{incolor}36}]:} \PY{n}{pd}\PY{o}{.}\PY{n}{plotting}\PY{o}{.}\PY{n}{scatter\PYZus{}matrix}\PY{p}{(}\PY{n}{dataset}\PY{p}{,} \PY{n}{figsize}\PY{o}{=}\PY{p}{(}\PY{l+m+mi}{15}\PY{p}{,}\PY{l+m+mi}{15}\PY{p}{)}\PY{p}{)}
         \PY{n}{plt}\PY{o}{.}\PY{n}{show}\PY{p}{(}\PY{p}{)}
\end{Verbatim}


    \begin{center}
    \adjustimage{max size={0.9\linewidth}{0.9\paperheight}}{output_14_0.png}
    \end{center}
    { \hspace*{\fill} \\}
    
    \subsection{Evaluate Some Algorithms}\label{evaluate-some-algorithms}

\begin{quote}
Now it is time to create some models of the data and estimate their
accuracy on unseen data.

Here is what we are going to cover in this step:

\begin{enumerate}
\def\labelenumi{\arabic{enumi}.}
\tightlist
\item
  Separate out a validation dataset.
\item
  Set-up the test harness to use 10-fold cross validation.
\item
  Build 5 different models to predict species from flower measurements
\item
  Select the best model.
\end{enumerate}
\end{quote}

    \subsubsection{Shape the data}\label{shape-the-data}

Remove the data from the pandas dataframe. Then split the names from the
measurements.

    \begin{Verbatim}[commandchars=\\\{\}]
{\color{incolor}In [{\color{incolor}37}]:} \PY{k+kn}{from} \PY{n+nn}{sklearn}\PY{n+nn}{.}\PY{n+nn}{metrics} \PY{k}{import} \PY{n}{classification\PYZus{}report}
         \PY{k+kn}{from} \PY{n+nn}{sklearn}\PY{n+nn}{.}\PY{n+nn}{metrics} \PY{k}{import} \PY{n}{confusion\PYZus{}matrix}
         \PY{k+kn}{from} \PY{n+nn}{sklearn}\PY{n+nn}{.}\PY{n+nn}{metrics} \PY{k}{import} \PY{n}{accuracy\PYZus{}score}
         \PY{k+kn}{from} \PY{n+nn}{sklearn}\PY{n+nn}{.}\PY{n+nn}{linear\PYZus{}model} \PY{k}{import} \PY{n}{LogisticRegression}
         \PY{k+kn}{from} \PY{n+nn}{sklearn}\PY{n+nn}{.}\PY{n+nn}{tree} \PY{k}{import} \PY{n}{DecisionTreeClassifier}
         \PY{k+kn}{from} \PY{n+nn}{sklearn}\PY{n+nn}{.}\PY{n+nn}{neighbors} \PY{k}{import} \PY{n}{KNeighborsClassifier}
         \PY{k+kn}{from} \PY{n+nn}{sklearn}\PY{n+nn}{.}\PY{n+nn}{discriminant\PYZus{}analysis} \PY{k}{import} \PY{n}{LinearDiscriminantAnalysis}
         \PY{k+kn}{from} \PY{n+nn}{sklearn}\PY{n+nn}{.}\PY{n+nn}{naive\PYZus{}bayes} \PY{k}{import} \PY{n}{GaussianNB}
         \PY{k+kn}{from} \PY{n+nn}{sklearn}\PY{n+nn}{.}\PY{n+nn}{svm} \PY{k}{import} \PY{n}{SVC}
\end{Verbatim}


    \begin{Verbatim}[commandchars=\\\{\}]
{\color{incolor}In [{\color{incolor}38}]:} \PY{n}{iris\PYZus{}array} \PY{o}{=} \PY{n}{dataset}\PY{o}{.}\PY{n}{values}
         
         \PY{c+c1}{\PYZsh{} Examine a single row.}
         \PY{c+c1}{\PYZsh{} Use a non\PYZhy{}crazy variable name.}
         \PY{n}{iris\PYZus{}array}\PY{p}{[}\PY{l+m+mi}{0}\PY{p}{]}
\end{Verbatim}


\begin{Verbatim}[commandchars=\\\{\}]
{\color{outcolor}Out[{\color{outcolor}38}]:} array([5.1, 3.5, 1.4, 0.2, 'setosa'], dtype=object)
\end{Verbatim}
            
    \begin{Verbatim}[commandchars=\\\{\}]
{\color{incolor}In [{\color{incolor}39}]:} \PY{c+c1}{\PYZsh{} Use sane variable names. Never use \PYZsq{}X\PYZsq{} or \PYZsq{}Y\PYZsq{} in cases like this.}
         \PY{n}{iris\PYZus{}data} \PY{o}{=} \PY{n}{iris\PYZus{}array}\PY{p}{[}\PY{p}{:}\PY{p}{,} \PY{l+m+mi}{0}\PY{p}{:}\PY{l+m+mi}{4}\PY{p}{]}
         \PY{n}{iris\PYZus{}names} \PY{o}{=} \PY{n}{iris\PYZus{}array}\PY{p}{[}\PY{p}{:}\PY{p}{,} \PY{l+m+mi}{4}\PY{p}{]}
\end{Verbatim}


    \begin{Verbatim}[commandchars=\\\{\}]
{\color{incolor}In [{\color{incolor}40}]:} \PY{c+c1}{\PYZsh{} Split array and variable assignment operations.}
         \PY{n}{validation\PYZus{}size} \PY{o}{=} \PY{l+m+mf}{0.20}
         \PY{n}{seed} \PY{o}{=} \PY{l+m+mi}{42}
         \PY{n}{scoring} \PY{o}{=} \PY{l+s+s1}{\PYZsq{}}\PY{l+s+s1}{accuracy}\PY{l+s+s1}{\PYZsq{}}
\end{Verbatim}


    \begin{Verbatim}[commandchars=\\\{\}]
{\color{incolor}In [{\color{incolor}41}]:} \PY{c+c1}{\PYZsh{} I do not leave off the package name\PYZhy{}space in most cases.}
         \PY{n}{iris\PYZus{}data\PYZus{}train}\PY{p}{,} \PYZbs{}
         \PY{n}{iris\PYZus{}data\PYZus{}validation}\PY{p}{,} \PYZbs{}
         \PY{n}{iris\PYZus{}name\PYZus{}train}\PY{p}{,} \PYZbs{}
         \PY{n}{iris\PYZus{}name\PYZus{}validation} \PY{o}{=} \PYZbs{}
         \PY{n}{sk}\PY{o}{.}\PY{n}{model\PYZus{}selection}\PY{o}{.}\PY{n}{train\PYZus{}test\PYZus{}split}\PY{p}{(}
             \PY{n}{iris\PYZus{}data}\PY{p}{,}
             \PY{n}{iris\PYZus{}names}\PY{p}{,}
             \PY{n}{test\PYZus{}size}\PY{o}{=}\PY{n}{validation\PYZus{}size}\PY{p}{,}
             \PY{n}{random\PYZus{}state}\PY{o}{=}\PY{n}{seed}\PY{p}{,}
         \PY{p}{)}
\end{Verbatim}


    \begin{Verbatim}[commandchars=\\\{\}]
{\color{incolor}In [{\color{incolor}49}]:} \PY{c+c1}{\PYZsh{} Spot Check Algorithms}
         \PY{n}{models} \PY{o}{=} \PY{p}{[}
             \PY{p}{(}\PY{l+s+s1}{\PYZsq{}}\PY{l+s+s1}{LR}\PY{l+s+s1}{\PYZsq{}}\PY{p}{,} \PY{n}{LogisticRegression}\PY{p}{(}\PY{p}{)}\PY{p}{)}\PY{p}{,}
             \PY{p}{(}\PY{l+s+s1}{\PYZsq{}}\PY{l+s+s1}{LDA}\PY{l+s+s1}{\PYZsq{}}\PY{p}{,} \PY{n}{LinearDiscriminantAnalysis}\PY{p}{(}\PY{p}{)}\PY{p}{)}\PY{p}{,}
             \PY{p}{(}\PY{l+s+s1}{\PYZsq{}}\PY{l+s+s1}{KNN}\PY{l+s+s1}{\PYZsq{}}\PY{p}{,} \PY{n}{KNeighborsClassifier}\PY{p}{(}\PY{p}{)}\PY{p}{)}\PY{p}{,}
             \PY{p}{(}\PY{l+s+s1}{\PYZsq{}}\PY{l+s+s1}{CART}\PY{l+s+s1}{\PYZsq{}}\PY{p}{,} \PY{n}{DecisionTreeClassifier}\PY{p}{(}\PY{p}{)}\PY{p}{)}\PY{p}{,}
             \PY{p}{(}\PY{l+s+s1}{\PYZsq{}}\PY{l+s+s1}{NB}\PY{l+s+s1}{\PYZsq{}}\PY{p}{,} \PY{n}{GaussianNB}\PY{p}{(}\PY{p}{)}\PY{p}{)}\PY{p}{,}
             \PY{p}{(}\PY{l+s+s1}{\PYZsq{}}\PY{l+s+s1}{SVM}\PY{l+s+s1}{\PYZsq{}}\PY{p}{,} \PY{n}{SVC}\PY{p}{(}\PY{p}{)}\PY{p}{)}\PY{p}{,}
         \PY{p}{]}
         
         \PY{c+c1}{\PYZsh{} evaluate each model in turn}
         \PY{n}{results} \PY{o}{=} \PY{p}{[}\PY{p}{]}
         \PY{n}{names} \PY{o}{=} \PY{p}{[}\PY{p}{]}
         \PY{k}{for} \PY{n}{name}\PY{p}{,} \PY{n}{model} \PY{o+ow}{in} \PY{n}{models}\PY{p}{:}
             
             \PY{n}{kfold} \PY{o}{=} \PY{n}{sk}\PY{o}{.}\PY{n}{model\PYZus{}selection}\PY{o}{.}\PY{n}{KFold}\PY{p}{(}\PY{n}{n\PYZus{}splits}\PY{o}{=}\PY{l+m+mi}{10}\PY{p}{,} \PY{n}{random\PYZus{}state}\PY{o}{=}\PY{n}{seed}\PY{p}{)}
             
             \PY{n}{cv\PYZus{}results} \PY{o}{=} \PY{n}{sk}\PY{o}{.}\PY{n}{model\PYZus{}selection}\PY{o}{.}\PY{n}{cross\PYZus{}val\PYZus{}score}\PY{p}{(}
                 \PY{n}{model}\PY{p}{,} 
                 \PY{n}{iris\PYZus{}data\PYZus{}train}\PY{p}{,} 
                 \PY{n}{iris\PYZus{}name\PYZus{}train}\PY{p}{,} 
                 \PY{n}{cv}\PY{o}{=}\PY{n}{kfold}\PY{p}{,} 
                 \PY{n}{scoring}\PY{o}{=}\PY{n}{scoring}
             \PY{p}{)}
             
             \PY{n}{results}\PY{o}{.}\PY{n}{append}\PY{p}{(}\PY{n}{cv\PYZus{}results}\PY{p}{)}
             \PY{n}{names}\PY{o}{.}\PY{n}{append}\PY{p}{(}\PY{n}{name}\PY{p}{)}
             
             \PY{n}{msg} \PY{o}{=} \PY{n}{f}\PY{l+s+s2}{\PYZdq{}}\PY{l+s+si}{\PYZob{}name\PYZcb{}}\PY{l+s+s2}{: }\PY{l+s+s2}{\PYZob{}}\PY{l+s+s2}{cv\PYZus{}results.mean()\PYZcb{} }\PY{l+s+s2}{\PYZob{}}\PY{l+s+s2}{cv\PYZus{}results.std()\PYZcb{}}\PY{l+s+s2}{\PYZdq{}} 
             \PY{n+nb}{print}\PY{p}{(}\PY{n}{msg}\PY{p}{)}
\end{Verbatim}


    \begin{Verbatim}[commandchars=\\\{\}]
LR: 0.95 0.04082482904638632
LDA: 0.975 0.03818813079129868
KNN: 0.9499999999999998 0.055277079839256664
CART: 0.9416666666666667 0.053359368645273735
NB: 0.9499999999999998 0.055277079839256664
SVM: 0.9583333333333333 0.04166666666666669

    \end{Verbatim}

    \begin{Verbatim}[commandchars=\\\{\}]
{\color{incolor}In [{\color{incolor}43}]:} \PY{n}{fig} \PY{o}{=} \PY{n}{plt}\PY{o}{.}\PY{n}{figure}\PY{p}{(}\PY{p}{)}
         \PY{n}{fig}\PY{o}{.}\PY{n}{suptitle}\PY{p}{(}\PY{l+s+s1}{\PYZsq{}}\PY{l+s+s1}{Algorithm Comparison}\PY{l+s+s1}{\PYZsq{}}\PY{p}{)}
         \PY{n}{ax} \PY{o}{=} \PY{n}{fig}\PY{o}{.}\PY{n}{add\PYZus{}subplot}\PY{p}{(}\PY{l+m+mi}{111}\PY{p}{)}
         \PY{n}{plt}\PY{o}{.}\PY{n}{boxplot}\PY{p}{(}\PY{n}{results}\PY{p}{)}
         \PY{n}{ax}\PY{o}{.}\PY{n}{set\PYZus{}xticklabels}\PY{p}{(}\PY{n}{names}\PY{p}{)}
         \PY{n}{plt}\PY{o}{.}\PY{n}{show}\PY{p}{(}\PY{p}{)}
\end{Verbatim}


    \begin{center}
    \adjustimage{max size={0.9\linewidth}{0.9\paperheight}}{output_23_0.png}
    \end{center}
    { \hspace*{\fill} \\}
    
    \begin{Verbatim}[commandchars=\\\{\}]
{\color{incolor}In [{\color{incolor}61}]:} \PY{c+c1}{\PYZsh{} Create a dataframe.}
         \PY{n}{result\PYZus{}df} \PY{o}{=} \PY{n}{pd}\PY{o}{.}\PY{n}{DataFrame}\PY{p}{(}
             \PY{n}{np}\PY{o}{.}\PY{n}{array}\PY{p}{(}\PY{n}{results}\PY{p}{)}\PY{o}{.}\PY{n}{T}\PY{p}{,}
             \PY{n}{columns}\PY{o}{=}\PY{n}{names}
         \PY{p}{)}
         \PY{n}{result\PYZus{}df}
\end{Verbatim}


\begin{Verbatim}[commandchars=\\\{\}]
{\color{outcolor}Out[{\color{outcolor}61}]:}          LR       LDA       KNN      CART        NB       SVM
         0  0.916667  1.000000  1.000000  1.000000  1.000000  1.000000
         1  0.916667  0.916667  0.916667  1.000000  0.916667  1.000000
         2  1.000000  1.000000  1.000000  1.000000  1.000000  1.000000
         3  0.916667  1.000000  1.000000  0.916667  1.000000  0.916667
         4  0.916667  0.916667  0.833333  0.833333  0.833333  0.916667
         5  0.916667  0.916667  0.916667  0.916667  0.916667  0.916667
         6  1.000000  1.000000  0.916667  0.916667  1.000000  0.916667
         7  1.000000  1.000000  1.000000  1.000000  1.000000  1.000000
         8  0.916667  1.000000  1.000000  0.916667  0.916667  1.000000
         9  1.000000  1.000000  0.916667  0.916667  0.916667  0.916667
\end{Verbatim}
            
    \begin{Verbatim}[commandchars=\\\{\}]
{\color{incolor}In [{\color{incolor}70}]:} \PY{n}{melted\PYZus{}result\PYZus{}df} \PY{o}{=} \PY{n}{pd}\PY{o}{.}\PY{n}{melt}\PY{p}{(}
             \PY{n}{result\PYZus{}df}\PY{p}{,}
             \PY{n}{var\PYZus{}name}\PY{o}{=}\PY{l+s+s1}{\PYZsq{}}\PY{l+s+s1}{method}\PY{l+s+s1}{\PYZsq{}}\PY{p}{,}
             \PY{n}{value\PYZus{}name}\PY{o}{=}\PY{l+s+s1}{\PYZsq{}}\PY{l+s+s1}{accuracy}\PY{l+s+s1}{\PYZsq{}}
         \PY{p}{)}
\end{Verbatim}


    \begin{Verbatim}[commandchars=\\\{\}]
{\color{incolor}In [{\color{incolor}86}]:} \PY{n}{fig} \PY{o}{=} \PY{n}{plt}\PY{o}{.}\PY{n}{figure}\PY{p}{(}\PY{p}{)}
         \PY{n}{fig}\PY{o}{.}\PY{n}{suptitle}\PY{p}{(}\PY{l+s+s1}{\PYZsq{}}\PY{l+s+s1}{Algorithm Comparison}\PY{l+s+s1}{\PYZsq{}}\PY{p}{)}
         \PY{n}{ax} \PY{o}{=} \PY{n}{fig}\PY{o}{.}\PY{n}{add\PYZus{}subplot}\PY{p}{(}\PY{l+m+mi}{111}\PY{p}{)}
         
         \PY{n}{sns}\PY{o}{.}\PY{n}{boxplot}\PY{p}{(}
             \PY{n}{data}\PY{o}{=}\PY{n}{melted\PYZus{}result\PYZus{}df}\PY{p}{,}
             \PY{n}{x}\PY{o}{=}\PY{l+s+s1}{\PYZsq{}}\PY{l+s+s1}{method}\PY{l+s+s1}{\PYZsq{}}\PY{p}{,}
             \PY{n}{y}\PY{o}{=}\PY{l+s+s1}{\PYZsq{}}\PY{l+s+s1}{accuracy}\PY{l+s+s1}{\PYZsq{}}\PY{p}{,}
             \PY{n}{ax}\PY{o}{=}\PY{n}{ax}\PY{p}{,}
             \PY{n}{color}\PY{o}{=}\PY{l+s+s1}{\PYZsq{}}\PY{l+s+s1}{w}\PY{l+s+s1}{\PYZsq{}}
         \PY{p}{)}
         
         
         \PY{n}{sns}\PY{o}{.}\PY{n}{swarmplot}\PY{p}{(}
             \PY{n}{data}\PY{o}{=}\PY{n}{melted\PYZus{}result\PYZus{}df}\PY{p}{,}
             \PY{n}{x}\PY{o}{=}\PY{l+s+s1}{\PYZsq{}}\PY{l+s+s1}{method}\PY{l+s+s1}{\PYZsq{}}\PY{p}{,}
             \PY{n}{y}\PY{o}{=}\PY{l+s+s1}{\PYZsq{}}\PY{l+s+s1}{accuracy}\PY{l+s+s1}{\PYZsq{}}\PY{p}{,}
             \PY{n}{ax}\PY{o}{=}\PY{n}{ax}\PY{p}{,}  \PY{c+c1}{\PYZsh{} Sometimes using default names gets confusing!}
         \PY{p}{)}
         
         
         \PY{n}{ax}\PY{o}{.}\PY{n}{set\PYZus{}ylim}\PY{p}{(}\PY{p}{[}\PY{l+m+mi}{0}\PY{p}{,} \PY{l+m+mf}{1.1}\PY{p}{]}\PY{p}{)}
\end{Verbatim}


\begin{Verbatim}[commandchars=\\\{\}]
{\color{outcolor}Out[{\color{outcolor}86}]:} (0, 1.1)
\end{Verbatim}
            
    \begin{center}
    \adjustimage{max size={0.9\linewidth}{0.9\paperheight}}{output_26_1.png}
    \end{center}
    { \hspace*{\fill} \\}
    
    \section{What is going on?}\label{what-is-going-on}

I will walk through an implementation of the k-nearest neighbors as
shown in \emph{Data Science from Scratch} by Joel Grus.

\begin{center}\rule{0.5\linewidth}{\linethickness}\end{center}

    \begin{Verbatim}[commandchars=\\\{\}]
{\color{incolor}In [{\color{incolor}93}]:} \PY{c+c1}{\PYZsh{} import some more packages.}
         \PY{k+kn}{import} \PY{n+nn}{random}
         \PY{k+kn}{import} \PY{n+nn}{collections}
\end{Verbatim}


    \begin{Verbatim}[commandchars=\\\{\}]
{\color{incolor}In [{\color{incolor}90}]:} \PY{k}{def} \PY{n+nf}{split\PYZus{}data}\PY{p}{(}\PY{n}{data}\PY{p}{,} \PY{n}{prob}\PY{p}{)}\PY{p}{:}
             \PY{l+s+sd}{\PYZdq{}\PYZdq{}\PYZdq{}}
         \PY{l+s+sd}{    Split data into fractions [prob, 1 \PYZhy{} prob]}
         \PY{l+s+sd}{    \PYZdq{}\PYZdq{}\PYZdq{}}
             
             \PY{n}{results} \PY{o}{=} \PY{p}{[}\PY{p}{]}\PY{p}{,} \PY{p}{[}\PY{p}{]}
             
             \PY{k}{for} \PY{n}{row} \PY{o+ow}{in} \PY{n}{data}\PY{p}{:}
                 \PY{n}{results}\PY{p}{[}\PY{l+m+mi}{0} \PY{k}{if} \PY{n}{random}\PY{o}{.}\PY{n}{random}\PY{p}{(}\PY{p}{)} \PY{o}{\PYZlt{}} \PY{n}{prob} \PY{k}{else} \PY{l+m+mi}{1}\PY{p}{]}\PY{o}{.}\PY{n}{append}\PY{p}{(}\PY{n}{row}\PY{p}{)}
                 
             \PY{k}{return} \PY{n}{results}
\end{Verbatim}


    \begin{Verbatim}[commandchars=\\\{\}]
{\color{incolor}In [{\color{incolor}92}]:} \PY{k}{def} \PY{n+nf}{train\PYZus{}test\PYZus{}split}\PY{p}{(}\PY{n}{input\PYZus{}vars}\PY{p}{,} \PY{n}{output\PYZus{}vars}\PY{p}{,} \PY{n}{test\PYZus{}pct}\PY{p}{)}\PY{p}{:}
             \PY{l+s+sd}{\PYZdq{}\PYZdq{}\PYZdq{}}
         \PY{l+s+sd}{    Splits a dataset and its answer set by the test\PYZus{}pct.}
         \PY{l+s+sd}{    \PYZdq{}\PYZdq{}\PYZdq{}}
             
             \PY{n}{data} \PY{o}{=} \PY{n+nb}{zip}\PY{p}{(}\PY{n}{input\PYZus{}vars}\PY{p}{,} \PY{n}{output\PYZus{}vars}\PY{p}{)}
             \PY{n}{train}\PY{p}{,} \PY{n}{test} \PY{o}{=} \PY{n}{split\PYZus{}data}\PY{p}{(}\PY{n}{data}\PY{p}{,} \PY{l+m+mi}{1} \PY{o}{\PYZhy{}} \PY{n}{test\PYZus{}pct}\PY{p}{)}
             
             \PY{c+c1}{\PYZsh{} What does this zip call do?}
             \PY{n}{input\PYZus{}train}\PY{p}{,} \PY{n}{output\PYZus{}train} \PY{o}{=} \PY{n+nb}{zip}\PY{p}{(}\PY{o}{*}\PY{n}{train}\PY{p}{)}
             \PY{n}{intput\PYZus{}test}\PY{p}{,} \PY{n}{output\PYZus{}test} \PY{o}{=} \PY{n+nb}{zip}\PY{p}{(}\PY{o}{*}\PY{n}{test}\PY{p}{)}
             
             \PY{k}{return} \PY{n}{input\PYZus{}train}\PY{p}{,} \PY{n}{intput\PYZus{}test}\PY{p}{,} \PY{n}{output\PYZus{}train}\PY{p}{,} \PY{n}{output\PYZus{}test}
\end{Verbatim}


    \subsection{k-Nearest Neighbors}\label{k-nearest-neighbors}

We need:

\begin{itemize}
\tightlist
\item
  Some notion of distance.
\item
  An assumption that the points that are close to one another are
  similar.
\end{itemize}

    \begin{Verbatim}[commandchars=\\\{\}]
{\color{incolor}In [{\color{incolor} }]:} \PY{k}{def} \PY{n+nf}{squared\PYZus{}distance}\PY{p}{(}\PY{p}{)}\PY{p}{:}
            \PY{k}{return} \PY{n}{sum\PYZus{}of\PYZus{}squares}\PY{p}{(}\PY{n}{vector\PYZus{}subtract}\PY{p}{(}\PY{n}{v}\PY{p}{,} \PY{n}{w}\PY{p}{)}\PY{p}{)}
\end{Verbatim}


    \begin{Verbatim}[commandchars=\\\{\}]
{\color{incolor}In [{\color{incolor}95}]:} \PY{k}{def} \PY{n+nf}{majority\PYZus{}vote}\PY{p}{(}\PY{n}{labels}\PY{p}{)}\PY{p}{:}
             \PY{l+s+sd}{\PYZdq{}\PYZdq{}\PYZdq{}Assumes labels are ordered from nearest to farthest.\PYZdq{}\PYZdq{}\PYZdq{}}
             
             \PY{n}{vote\PYZus{}counts} \PY{o}{=} \PY{n}{collections}\PY{o}{.}\PY{n}{Counter}\PY{p}{(}\PY{n}{labels}\PY{p}{)}
             
             \PY{n}{winner}\PY{p}{,} \PY{n}{winner\PYZus{}count} \PY{o}{=} \PY{n}{vote\PYZus{}counts}\PY{o}{.}\PY{n}{most\PYZus{}common}\PY{p}{(}\PY{l+m+mi}{1}\PY{p}{)}\PY{p}{[}\PY{l+m+mi}{0}\PY{p}{]}
             
             \PY{n}{num\PYZus{}winners} \PY{o}{=} \PY{n+nb}{len}\PY{p}{(}\PY{p}{[}\PY{n}{count} \PY{k}{for} \PY{n}{count} \PY{o+ow}{in} \PY{n}{vote\PYZus{}counts}\PY{o}{.}\PY{n}{values}\PY{p}{(}\PY{p}{)}
                                \PY{k}{if} \PY{n}{count} \PY{o}{==} \PY{n}{winner\PYZus{}count}\PY{p}{]}\PY{p}{)}
             
             \PY{k}{if} \PY{n}{num\PYZus{}winners} \PY{o}{==} \PY{l+m+mi}{1}\PY{p}{:}
                 \PY{k}{return} \PY{n}{winner}
             \PY{k}{else}\PY{p}{:}
                 \PY{k}{return} \PY{n}{majority\PYZus{}vote}\PY{p}{(}\PY{n}{labels}\PY{p}{[}\PY{p}{:}\PY{o}{\PYZhy{}}\PY{l+m+mi}{1}\PY{p}{]}\PY{p}{)}
\end{Verbatim}


    \begin{Verbatim}[commandchars=\\\{\}]
{\color{incolor}In [{\color{incolor}97}]:} \PY{k}{def} \PY{n+nf}{knn\PYZus{}classify}\PY{p}{(}\PY{n}{k}\PY{p}{,} \PY{n}{labeled\PYZus{}points}\PY{p}{,} \PY{n}{new\PYZus{}point}\PY{p}{)}\PY{p}{:}
             \PY{l+s+sd}{\PYZdq{}\PYZdq{}\PYZdq{}}
         \PY{l+s+sd}{    Each labeled point should be a pair (point, label)}
         \PY{l+s+sd}{    \PYZdq{}\PYZdq{}\PYZdq{}}
             
             \PY{c+c1}{\PYZsh{} Order the labeled points from nearest to farthest.}
             \PY{n}{by\PYZus{}distance} \PY{o}{=} \PY{n+nb}{sorted}\PY{p}{(}\PY{n}{labeled\PYZus{}points}\PY{p}{,} \PY{n}{key}\PY{o}{=}\PY{k}{lambda} \PY{n}{point}\PY{p}{,} \PY{n}{\PYZus{}}\PY{p}{:} \PY{n}{distance}\PY{p}{(}\PY{n}{point}\PY{p}{,} \PY{n}{new\PYZus{}point}\PY{p}{)}\PY{p}{)}
             
             \PY{c+c1}{\PYZsh{} Find the labels for the k closest.}
             \PY{n}{k\PYZus{}nearest\PYZus{}labels} \PY{o}{=} \PY{p}{[}\PY{n}{label} \PY{k}{for} \PY{n}{unused}\PY{p}{,} \PY{n}{label} \PY{o+ow}{in} \PY{n}{by\PYZus{}distance}\PY{p}{[}\PY{p}{:}\PY{n}{k}\PY{p}{]}\PY{p}{]}
             
             \PY{c+c1}{\PYZsh{} Vote.}
             \PY{k}{return} \PY{n}{majority\PYZus{}vote}\PY{p}{(}\PY{n}{k\PYZus{}nearest\PYZus{}labels}\PY{p}{)}
\end{Verbatim}



    % Add a bibliography block to the postdoc
    
    
    
    \end{document}
